\documentclass[11pt,a4paper,twoside,openright]{report}
\usepackage[latin1]{inputenc}
\usepackage[T1]{fontenc}
\usepackage[colorlinks=true]{hyperref}
\usepackage[usenames]{color}
\usepackage{fancyvrb,varioref}

\pagestyle{headings}
% \pagecolor[rgb]{0.134,0.40,0.40}

\begin{document}

% \color{white}
\title{Binc IMAP manual (DRAFT)}
\author{Andreas Aardal Hanssen}

\maketitle
\tableofcontents

%----------------------------------------------------------------
%----------------------------------------------------------------
%----------------------------------------------------------------
\chapter[Introducing Binc IMAP]{Introduction}

\begin{quote}
``As an alternative to existing similar IMAP servers, Binc IMAP
strives to be very easy to use, but robust, stable and secure.''
\end{quote}
Welcome to Binc IMAP, a project started December 2003 by Andreas
Aardal Hanssen. Binc IMAP is an open source IMAP project which differs
from existing IMAP projects. Here is a list of the project goals:

\begin{enumerate}
\item Binc IMAP will always have a helpful, hospitable community.
  \begin {itemize}
  \item Although it is expected that users of Binc IMAP do their
  homework before posting to the mailing list, the server author and
  community of the Binc IMAP project will be friendly and will
  approach everyone with respect. The same behavior is expected from
  those who post to the list.

  \item There will be no RTFM\footnote{According to acronymfinder.com,
  RTFM stands, among many other suggestions, for ``Read The Fscking
  Manual'' (fsck is a Unix filesystem checker)} on the project's
  mailing list. Flaming and personal insults on the project's mailing
  list will result in banning of the originator.

  \item The community is encouraged to pay back to the project's
  contributors by sharing their own experience and contributions to
  Binc IMAP through the GPL license.
  \end{itemize}

\item Binc IMAP will not compete with other IMAP projects
  \begin{itemize}
  \item Under no circumstance will this project be in market driven
  competition with other IMAP servers.

  \item Binc IMAP will first and foremost be a quality driven project.

  \item This project is meant to influence the community of authors of
  similar network protocols and servers, and hopes to increase the
  general quality of software that is used all over the globe and
  beyond.
  \end{itemize}

\item Binc IMAP provides security through good design
  \begin{itemize}
  \item A well designed server is less exposed to bugs than a poorly
  designed server.

  \item The server will strive to use every kind of security enhancing
  feature, while keeping the implementation details as good and simple
  as possible.

  \item The source is open and downloadable. Potential bugs and/or
  nasty pieces of code are easily uncovered when the whole community
  is able to study every line of code in detail. Bugs should always be
  reported to the project's mailing list.
  \end{itemize}

\item Binc IMAP is modular
  \begin{itemize}
  \item Where possible and practical from both a usage and design
    stand point, modules will be seperated and replaceable through
    \textit{pluggable extension support}. Examples of future
    replaceables and enhancements include:
    \begin{itemize}
    \item authentication modules (shadow, PAM, LDAP)
    \item search modules
    \item protocol extensions modules (namespaces, ACL, shared folders)
    \item mailbox formats (Maildir, mbox, MySQL, POP3 proxy)
    \end{itemize}

  \item With a modular and good object oriented design, it will be
  easy to quickly understand what every method and function does. This
  will increase third party developers' ability to write extensions
  and modifications fast.

  \end{itemize}

\item Binc IMAP favors quality over quantity
  
  \begin{itemize}
  \item Binc IMAP's releases are milestones. We strive for perfection.

  \item Work on improving the existing design and extensibility will
  always go ahead of adding new features.

  \item Through extensive module support, the community is encouraged
  to contribute to the adding and testing of new features.

  \item Core design and implementation will always focus on quality.
  \end{itemize}

\end{enumerate}

\textbf{This document serves as the server's main documentation.}

%----------------------------------------------------------------
%----------------------------------------------------------------
%----------------------------------------------------------------
\chapter[Installing Binc IMAP]{Installation}

The installation procedure for Binc IMAP is designed to be quick and
easy, and in most cases you will get what you want at your first
attempt. The source code is designed to be able to compile on most
UNIX-like platforms.

If you experience problems with compiling and installing this package
on your own platform, don't hesitate to post your problem to the
project's mailing list.

%----------------------------------------------------------------
\section{Downloading the package}

Binc IMAP is available for a number of platforms and distributions.
Download the package most suitable for your distribution from one
of the following locations:

\begin{itemize}

\item Red Hat Linux:
  \begin{itemize}
  \item \url{http://www.bincimap.org/}
  \item \url{http://www.bincimap.andreas.hanssen.name/}
  \item \url{http://bincimap.argonsoft.de/}
  \end{itemize}

\item Mandrake Linux:
  \begin{itemize}
  \item \url{ftp://sunsite.uio.no/linux/Mandrake/Mandrake-devel/contrib}
  \end{itemize}

\item Debian Linux:
  \begin{itemize}
  \item \url{http://packages.debian.org/testing/mail/bincimap.html}
  \item \url{http://packages.debian.org/unstable/mail/bincimap.html}
  \end{itemize}

\item FreeBSD:
  \begin{itemize}
  \item \url{http://www.freebsd.org/cgi/ports.cgi?query=bincimap&stype=all}
  \end{itemize}

\item Other (tarball):
  \begin{itemize}
  \item \url{http://www.bincimap.org/}
  \item \url{http://www.bincimap.andreas.hanssen.name/}
  \item \url{http://bincimap.argonsoft.de/}
  \end{itemize}

\end{itemize}

%----------------------------------------------------------------
\section{Building from the tarball}
\label{buildfromtarball}

This section describes how you can build the Binc IMAP server using
the compressed tar archive (\textit{tarball}). This approach should
work on all supported platforms.

The tarball creates a seperate directory in which it installs all
source files. The installation procedure is typical for open source C
and C++ software that uses the \texttt{autoconf} and \texttt{automake}
tools. First, unpack the tarball:

\begin{Verbatim}
# zcat bincimap-1.2.1.tar.gz | tar xvf -
<outputs a list of files>
\end{Verbatim}
Now, enter the directory that was created. In this directory, you will
find the \texttt{configure} script.

\begin{Verbatim}
# cd bincimap-1.2.1
# ls configure
configure
#
\end{Verbatim}
Run the configure script with all options you find appropriate. The
\texttt{configure} options are listed in
section\vref{configureoptions}. Most users will be ok with these
common options:

\begin{Verbatim}
# ./configure --prefix=/opt/bincimap --sysconfdir=/etc/opt/bincimap
\end{Verbatim}
The \texttt{configure} script will produce many lines of output, and
finally, if all goes well, it will create all the \texttt{Makefile}s
we need to build this projectm We can now run \texttt{make}:

\begin{Verbatim}
# make
\end{Verbatim}
If the compile finishes with no problems, you can install your new
software:

\begin{Verbatim}
# make install
\end{Verbatim}
If the \texttt{prefix} option is \texttt{/opt/bincimap}, \texttt{make
install} will create the following files:

\begin{Verbatim}
/opt/bincimap/bin/bincimap-up              # Startup stub
/opt/bincimap/bin/bincimapd                # Main server
/opt/bincimap/man/man1/bincimap-up.1       # Man page
/opt/bincimap/man/man1/bincimapd.1         # Man page
/opt/bincimap/man/man5/bincimap.conf.5     # Man page
\end{Verbatim}
The \texttt{sysconfdir} option tells Binc IMAP where to install the
configuration files. If \texttt{sysconfdir} is set to
\texttt{/etc/opt/bincimap}, \texttt{make install} will create the
following files:

\begin{Verbatim}
/etc/opt/bincimap/bincimap.conf            # Main conf file
/etc/opt/bincimap/xinetd/imap              # xinetd conf file
/etc/opt/bincimap/xinetd/imaps             # xinetd conf file
/etc/opt/bincimap/service/imap/run         # service conf file
/etc/opt/bincimap/service/imap/log/run     # service conf file
/etc/opt/bincimap/service/imaps/run        # service conf file
/etc/opt/bincimap/service/imaps/log/run    # service conf file
\end{Verbatim}
You are now ready to configure Binc IMAP. Read more about this in
section\vref{configuration}.

%- - - - - - - - - - - - - - - - - - - - - - - - - - - - - - - - 
\subsection{Options to configure}
\label{configureoptions}

\begin{itemize}

\item \Verb@--prefix=<dir>@\\
  Where to install binary files and man pages. Binary files
  will be installed under \Verb@<dir>/bin@ and man pages go
  under \Verb@<dir>@/man.

\item \Verb@--sysconfdir=<dir>@\\
  Where to install configuration files. The main conf file
  goes under \Verb@<dir>/bincimap.conf@ and other conf files
  go under \Verb@<dir>/<path>/...@, depending on the type
  of conf files.

  \textbf{Note:} The bundled service files
  must \textbf{not} be stored on an NFS partition or any partition
  that does not support file locking.

\item \Verb@--localstatedir=<dir>@\\
  Where log directories go for multilog logging. Directories
  \texttt{log/bincimap} and \texttt{log/bincimap-ssl} will be created
  under this prefix.

\item \Verb@--disable-ssl@\\
  Compile Binc IMAP with no SSL code, and without the OpenSSL
  requirement. Also removes the STARTTLS capability.

\item \Verb@--enable-ssl@ (default)\\
  Compile Binc IMAP with SSL code, and with the OpenSSL
  requirement. Enables the STARTTLS capability.

\item \Verb@--enable-static@\\
  Compile Binc IMAP with static linkage.

\item \Verb@--without-optimization@\\
  Compile Binc IMAP with -O0 (no optimization).

\end{itemize}

\section{Building from the src.rpm}

The \texttt{src.rpm} packages are designed for the Red Hat Linux
distribution. If you are not running Red Hat Linux or similar
distributions and this procedure fails, then you should build Binc
IMAP from the tarball (section\vref{buildfromtarball}).

\begin{itemize}

\item Make sure you have the \texttt{rpmbuild} program installed. In
Red Hat, it is a part of the \texttt{rpm-build} package.

\item As root\footnote{It is possible to build src.rpm packages as
non-root. There is info available on http://www.rpm.org/
about this.}, run the following command:

\begin{Verbatim}
rpmbuild --rebuild bincimap-1.2.1-1.src.rpm
\end{Verbatim}

\item If the command succeeds, you will have generated a binary rpm
package that you can install.

\begin{Verbatim}
rpm -i bincimap-1.2.1-1.i386.rpm
\end{Verbatim}

\end{itemize}

\section{After building and installing the package}

When the service has been installed successfully, you should take your
time to configure it properly. Please refer to
section\vref{configuration}.

%----------------------------------------------------------------
%----------------------------------------------------------------
%----------------------------------------------------------------
\chapter[Configuring Binc IMAP]{Configuration}
\label{configuration}

With most services, the hardest thing to get right is the
configuration file. With Binc IMAP, this should not be a problem for
the administrator. There are few options to set, and the package also
provides a default configuration file whose default settings are fine
for most uses.

This section will first give an overview of the format of Binc IMAP's
configuration file, and it will then explain every option that you
can set.

\section{Introduction to \texttt{Binc::Storage}}

Binc IMAP uses a special file format, called \texttt{Binc::Storage}
after its C++ class name. This file format is human readable and
editable, it's fast to parse programmatically, and it has very little
overhead when representing data. It also groups options together in
sections, supports comments (c-style) and arbitrary whitespace usage.

The structure of the file is as follows:

\begin{enumerate}
\item Aliases
\item Section1
  \begin{enumerate}
    \item key1=value1
    \item key2=value2
    \item ...
  \end{enumerate}
\item Section2
  \begin{enumerate}
    \item key1=value1
    \item key2=value2
    \item ...
  \end{enumerate}
\item SectionN...
\end{enumerate}
Moving on to the actual format, the file starts with an optional list
of aliases. These aliases start with a question mark (?), followed by
the alias, a colon (:), the replacement text, and finally a semicolon
(;). This example defines the alias "m" to represent the text
"myfirstsection":

\begin{Verbatim}
?m:MyFirstSection;
\end{Verbatim}
Then comes sections. A section starts with an identifier --- the
section name. The name contains only alphanumeric (a to z, A to Z and
0-9) characters, and it's case sensitive (``hello'' and ``HELLO''
could be two different sections). The section content itself is
enclosed in braces, as in the following example.

\begin{Verbatim}
MyFirstSection {

}
\end{Verbatim}
Inside a section we have a comma seperated list of key=value pairs.
The keys and values consist of a sequence of one or more alphanumeric
words. The value can also contain quoted strings:

\begin{Verbatim}
MyFirstSection {
    certificate path = "/etc/path/certificate.txt",
    optional argument = yes
}
\end{Verbatim}
The aliases from the start of the configuration file are useful if we
wish to appreviate long words in a key or section name in order to
save space. Normally this is only used in generated files, but they
can also make a manually edited conf file more compact. Here we have
replaced ``MyFirstSection'' with its alias ``m'':

\begin{Verbatim}
m {
    certificate path = "/etc/path/certificate.txt",
    optional argument = "yes"
}
\end{Verbatim}
When accessed by Binc IMAP, the two sections are treated equally, as
if there was no alias subsitution.

The following sections describe the \texttt{Binc::Storage} sections
that Binc IMAP recognizes.

\section{Sections in \texttt{bincimap.conf}}

Now followes a brief description of all recognized sections in
\texttt{bincimap.conf}.

%- - - - - - - - - - - - - - - - - - - - - - - - - - - - - - - - 
\subsection{Authentication}

The authentication details in Binc IMAP are set in a section called \texttt{Authentication}:

\begin{Verbatim}
Authentication {

}
\end{Verbatim}
The recognized options are:

\begin{itemize}

\item \texttt{allow plain auth in non ssl = yes{|}no}
  \begin{itemize}
    \item[\texttt{yes}] Clients are allowed to authenticate using
      plain text passwords when transport is not encrypted with SSL.
    \item[\texttt{no}] Clients are \textit{not} allowed to
      authenticate using plain text passwords when transport is not
      encrypted with SSL.
  \end{itemize}

\item \texttt{auth penalty = \Verb@<n>@}
  \begin{itemize}
    \item [] If a user tries to authenticate but the password or
      username is wrong, the server will sleep for \Verb@<n>@ seconds
      before allowing the user to try again.
  \end{itemize}
\end{itemize}
This is an example of an \texttt{Authentication} section in
\texttt{bincimap.conf}:

\begin{Verbatim}
Authentication {
    allow plain auth in non ssl = no,
    auth penalty = 4
}
\end{Verbatim}

%- - - - - - - - - - - - - - - - - - - - - - - - - - - - - - - - 
\subsection{Logging}

The logging details in Binc IMAP are set in a section called \texttt{Log}:

\begin{Verbatim}
Log {

}
\end{Verbatim}
The recognized options are:

\begin{itemize}

\item \texttt{type = syslog{|}multilog}
  \begin{itemize}
  \item [\texttt{syslog}] Binc IMAP will log using syslog.
  \item [\texttt{multilog}] Binc IMAP will log to \textit{stderr},
    which is the default input for \texttt{multilog}.
  \end{itemize}

\item \texttt{environment ip variable = <var>}
  \begin{itemize}
  \item [] If configured to log using \texttt{multilog}, Binc IMAP
    attempts to find the remote host's IP address in the environment
    variables.  This option allows the administrator to suggest which
    variable the IP address is stored in.
  \end{itemize}

\item \texttt{syslog facility = <facility>}
  \begin{itemize}
    \item [] If Binc IMAP is configured to log using syslog, it will
      log using the syslog facility described by the alias
      \Verb@<facility>@.
  \end{itemize}

\item \texttt{syslog facility number = <facilitynr>}
  \begin{itemize}
    \item [] If Binc IMAP is configured to log using syslog, it will
      log using the syslog facility number \Verb@<facilitynr>@. This
      is typically used if the textual facility name is not recognized
      by Binc IMAP.
  \end{itemize}
\end{itemize}
This is an example of a \texttt{Log} section in
\texttt{bincimap.conf}:

\begin{Verbatim}
Log {
    type = syslog,
    syslog facility = LOG_DAEMON
}
\end{Verbatim}

%- - - - - - - - - - - - - - - - - - - - - - - - - - - - - - - - 
\subsection{Security}

The security details in Binc IMAP are set in a section called \texttt{Security}:

\begin{Verbatim}
Security {

}
\end{Verbatim}
The recognized options are:

\begin{itemize}
\item \texttt{jail path = <path>}
  \begin{itemize}
    \item [] Binc IMAP's invocation stub, \texttt{bincimap-up}, enters
      a ``chroot jail'' path after starting the main server. The path
      is given in this setting and it must exist for Binc IMAP to
      function properly.
  \end{itemize}
\item \texttt{jail user = <userid>}
  \begin{itemize}
    \item [] Binc IMAP's invocation stub is by default run as the
      privileged user \texttt{root}. Before the stub enters its
      ``chroot jail'', it changes to the \Verb@<userid>@ user, which
      should be a user with no privileges.
  \end{itemize}
\item \texttt{jail group = <groupid>}
  \begin{itemize}
    \item [] As with \texttt{jail user}, but sets the \textit{group}
      that the invocation stub changes to before entering the jail.
  \end{itemize}
\end{itemize}
This is an example of a \texttt{Security} section in
\texttt{bincimap.conf}:

\begin{Verbatim}
Security {
    jail path = "/opt/bincimap/bin",
    jail user = nobody,
    jail group = nobody
}
\end{Verbatim}

%- - - - - - - - - - - - - - - - - - - - - - - - - - - - - - - - 
\subsection{Mailbox settings}

The mailbox settings in Binc IMAP are set in a section called \texttt{Mailbox}:

\begin{Verbatim}
Mailbox {

}
\end{Verbatim}
The recognized options are:

\begin{itemize}
\item \texttt{type = <type>}
  \begin{itemize}
    \item [] Sets the type of mailbox that Binc IMAP should
      use as default. Currently, the only supported format is
      \texttt{Maildir}.
  \end{itemize}
\item \texttt{depot = <depottype>}
  \begin{itemize}
    \item [] Sets the type of depot that Binc IMAP should use. The
      depot decides how the contents of an IMAP user's mail path is
      interpreted. Currenty, the two supported depot types are
      \texttt{IMAPdir} and \texttt{Maildir++}..
  \end{itemize}
\item \texttt{path = <path>}
  \begin{itemize}
    \item [] When a user authenticates, Binc IMAP can optionally
      change to a seperate mail directory within the user's home area.
      This directory is set with the \texttt{path} setting. For mail
      storage configurations where there is no seperate area for mail,
      set \texttt{path=""}.
  \end{itemize}
\item \texttt{auto create inbox = yes{|}no}
  \begin{itemize}
    \item [yes] Binc IMAP will create the INBOX mailbox if the IMAP
      user does not already have one.
    \item [no] The server will \textit{not} create INBOX when the user
      logs in for the first time.
  \end{itemize}
\item \texttt{auto subscribe mailboxes = <mailboxlist>}
  \begin{itemize}
    \item [] When logging onto the IMAP service for the first time,
      the user will automatically be subscribed to the mailboxes in
      the comma seperated list \Verb@<mailboxlist>@. The content of
      the list if for example "INBOX,Trash,Drafts,Sent".
  \end{itemize}
\item \texttt{umask = <number>}
  \begin{itemize}
    \item [] All operations in a user's mail directory will be
      performed using this umask.
  \end{itemize}


\end{itemize}
This is an example of a \texttt{Mailbox} section in
\texttt{bincimap.conf}:

\begin{Verbatim}
Mailbox {
    path = "Maildir",
    type = "Maildir",
    depot = "IMAPdir",
    auto create inbox = yes,
    auto subscribe mailboxes = "INBOX,Work,Work/Sent",
    umask = 0777
}
\end{Verbatim}

%- - - - - - - - - - - - - - - - - - - - - - - - - - - - - - - - 
\subsection{Session}

The session details in Binc IMAP are set in a section called \texttt{Session}:

\begin{Verbatim}
Session {

}
\end{Verbatim}
The recognized options are:

\begin{itemize}

\item \texttt{auth timeout = <seconds>}
  \begin{itemize}
    \item [] Before authenticating, Binc IMAP will allow clients to
      idle (stay connected with no activity) for \Verb@<second>@
      seconds before automatically closing the IMAP session.
  \end{itemize}

\item \texttt{idle timeout = <seconds>}
  \begin{itemize}
    \item [] After authenticating, Binc IMAP will allow clients to
      idle (stay connected with no activity) for \Verb@<second>@
      seconds before automatically closing the IMAP session.
  \end{itemize}

\item \texttt{transfer timeout = <seconds>}
  \begin{itemize}
    \item [] Every chunk of data that is sent by Binc IMAP to the IMAP
      client has \Verb@<seconds>@ seconds to complete. If this timeout
      is exceeded, the IMAP session is closed.
  \end{itemize}

\item \texttt{transfer buffer size = <nrofbytes>}
  \begin{itemize}
    \item [] To save network roundtrips, Binc IMAP buffers up to
      \Verb@<nrofbytes>@ bytes in memory before it is shipped off to
      the client. Greatly affects performance on high latency links.
  \end{itemize}
\end{itemize}


This is an example of a \texttt{Session} section in
\texttt{bincimap.conf}:

\begin{Verbatim}
Session {
  auth timeout = 30,
  idle timeout = 1800,
  transfer timeout = 1200,
  transfer buffer size = 1024
}
\end{Verbatim}

%- - - - - - - - - - - - - - - - - - - - - - - - - - - - - - - - 
\subsection{SSL/TLS}

The SSL details in Binc IMAP are set in a section called \texttt{SSL}:

\begin{Verbatim}
SSL {

}
\end{Verbatim}
The recognized options are:

\begin{itemize}

\item \texttt{pem file = <file>}
  \begin{itemize}
    \item [] Binc IMAP reads a PEM encoded SSL certificate from this
      location. This file can often be generated with scripts under
      \texttt{/usr/share/ssl/certs}.
  \end{itemize}

\item \texttt{ca file = <file>}
  \begin{itemize}
    \item [] Gives Binc IMAP a file that contains certificate
      authorities.  The purpose of this file is specific to the SSL
      protocol and it is used in the SSL hand shake.
  \end{itemize}

\item \texttt{ca path = <path>}
  \begin{itemize}
    \item [] Gives Binc IMAP a path in which the SSL library can
      search for files containing certificate authorities.  The
      purpose of this path is specific to the SSL protocol and it is
      used in the SSL hand shake.
  \end{itemize}

\item \texttt{cipher list = <ciphers>}
  \begin{itemize}
    \item [] The string provided in \Verb@<ciphers>@ informs the SSL
      library of which ciphers are available.
  \end{itemize}

\item \texttt{verify peer = yes{|}no}
  \begin{itemize}
    \item [] Informs Binc IMAP wether or not the server should attempt
      to verify the peer's certificate.
  \end{itemize}

\end{itemize}
This is an example of a \texttt{SSL} section in
\texttt{bincimap.conf}:

\begin{Verbatim}
SSL {
  pem file = "/usr/share/ssl/certs/server.pem",
  ca file = "",
  ca path = "",
  cipher list = "!ADH:RC4+RSA:HIGH:MEDIUM:LOW:EXP:+SSLv2:+EXP",
  verify peer = no
}
\end{Verbatim}

%----------------------------------------------------------------
%----------------------------------------------------------------
%----------------------------------------------------------------
\chapter{The Binc IMAP Depot}

The main role of an IMAP server is to give email clients an interface
through which they can authenticate and then access email and
mailboxes located on a (remote) mail server. The mails are stored on
the mail servers in a special structure, and this structure varies
from system to system.

Most often we distinguish between the mailbox \textit{structure} (or
\textit{hierarchy structure}) and its \textit{format}. The structure
determines how the different mailboxes and submailboxes are stored,
and the format decides how each and every email is stored within one
mailbox.

Binc IMAP uses the generic container object \textit{Depot} to describe
the map between the hierarchy structure and its translation to
selectable mailboxes in IMAP. The Depot has two specializations: one
for IMAPdir and one for Maildir++.

%----------------------------------------------------------------
\section{Maildir++}

Courier-IMAP defines the hierarchy structure Maildir++, which provides
a way for existing Maildir users (Maildir is a mailbox format) to have
multiple mailboxes and submailboxes inside the directory that contains
the default mailbox ``INBOX'' (which often resides in \Verb@~/Maildir@
for each UNIX user).

Read more about the Maildir++ format at the following location:

\begin{itemize}
\item [] \url{http://www.inter7.com/courierimap/README.maildirquota.html}
\end{itemize}

Binc IMAP supports this mailbox structure with a few limitations and a
few enhancements:

\begin{itemize}
\item The \textit{maildirfolder} file is not created inside each
Maildir submailbox. The reason for this is that this only works with
mailbox formats that store a mailbox inside a directory, such as
Maildir.
\item Maildir++ quotas are not supported
\item No Maildir++ restrictions to mailbox names apply (such as
\texttt{.Trash})
\item Mailboxes inside a Maildir++ structure can be of any format, not
just Maildir.
\end{itemize}

%----------------------------------------------------------------
\section{IMAPdir}

IMAPdir is Binc IMAP's native mailbox structure. It is open and usable
for most existing local mail clients. For a full description, look up
the following web site, or search for the
\texttt{bincimap-imapdir.html} file in the packaged documentation.

\begin{itemize}
\item [] \url{http://www.bincimap.org/bincimap-imapdir.html}
\end{itemize}

IMAPdir does not extend/change any mailbox formats; it merely defines
a way to describe mailboxes and submailboxes in a way that is suitable
for an IMAP server, with as few restrictions as necessary.


\end{document}

